\section{Analisi degli apparati di rete e mezzi trasmissivi}
Per quanto riguarda gli apparati di rete si andranno ad utilizzare schede di rete, router, AP e switch.\\
\hspace{24pt}La scheda di rete è un dispositivo elettronico installato all'interno di un host che permette il collegamento tra l'host e il cavo, che collega i vari nodi. Una scheda di rete può essere wireless, ossia non ha bisogno di un collegamento fisico via cavo, bensì comunica con l'AP tramite onde radio (etere).\\
\hspace{24pt}Il router è un dispositivo che permette la connessione tra due reti, in particolare una rete LAN e Internet. In questo progetto mette in comunicazione la rete della scuola con la rete Internet.\\
\hspace{24pt}Lo switch è un dispositivo che collega insieme altri dispositivi. Lo switch, rispetto all'hub, gestisce in modo più efficiente il trasporto dei dati perché inoltra il pacchetto ricevuto soltanto al destinatario.\\
\hspace{24pt}L'AP è un particolare tipo di switch che permette di collegare dispositivi possedenti una scheda di rete wireless, onde evitare di utilizzare troppi cavi. Un AP può essere pubblico o protetto. Se è pubblico non ha bisogno di una chiave d'accesso, se è protetto allora può utilizzare diversi sistemi di sicurezza. Il sistema WEP\footnote{Wired Equivalent Privacy, parte dello standard IEEE 802.11. Questo sistema di crittografia delle reti Wi-Fi fu introdotto per evitare il furto dei dati wireless da parte degli hacker.} prevede la presenza di una chiave per connettersi all'AP. I dati verranno criptati tramite questa chiave, in modo da renderli leggibili soltanto da chi è in possesso della chiave. Un'altro sistema di crittografia è quello di WPA/WPA2\footnote{Wi-Fi Protected Access}, più moderno e più sicuro del WEP.\\
\hspace{24pt}Per il cablaggio verticale saranno presenti dorsali di edificio che collegheranno il centro stella di edificio con gli switch di piano. Da qui verranno effettuati i collegamenti alle prese utente, che fanno parte del cablaggio orizzontale.