\section{Configurazione DNS}
\hspace{24pt}L'ospedale ha bisogno di un server DNS\footnote{Dynamic Name Service} per l'accesso al server web tramite stringa 
alfanumerica facilmente memorizzabile. Per fare ciò, bisogna abilitare il servizio DNS sul server incaricato ed 
aggiungere un Record contenente nome del sito e l'indirizzo IP del server web, in questo caso rispettivamente 
\mintinline{batch}{portaleinterno.it} e 192.168.0.3. In questo modo tutti gli host connessi alla rete possono accedere 
al server web digitando "portaleinterno.it" sul web browser anziché l'indirizzo ip numerico.