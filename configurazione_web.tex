\section{Configurazione server web}
\hspace{24pt}L'ospedale ha bisogno di un server web al quale i medici di reparto possono accedere per la gestione dei dati di un 
paziente. Per permettere ciò bisogna abilitare il servizio HTTP\footnote{Hyper-Text Transport Protocol: un protocollo 
per il trasferimento di informazioni tra server e client} sul server incaricato. Dopodiché bisogna assegnare un 
indirizzo IP statico, in questo caso 192.168.0.3. Una volta abilitato il servizio, si caricano i file del sito sul 
server web in modo da renderlo accessibile a tutti gli host collegati alla rete. 