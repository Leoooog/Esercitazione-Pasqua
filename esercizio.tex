Un ospedale vuole innovare la sua infrastruttura tecnologica per realizzare servizi interni. L’ospedale è
composto da 4 reparti distribuiti su due piani(ogni reparto si sviluppa su un unico piano). Ogni medico di
reparto, dopo avere visitato un paziente, può collegarsi in modalità wireless ad un server web interno,
dislocato in un locale tecnico posto al piano seminterrato, per registrare le informazioni medico-sanitarie
dei pazienti visitati oppure visualizzare la scheda del paziente. Nel locale tecnico è presente oltre al server
web anche un server DHCP per l’assegnazione automatica degli indirizzi IP ai vari dispositivi e un server
DNS. In ogni reparto è presente anche una saletta medica con due postazioni fisse e una stampante.
Al piano seminterrato è presente anche la farmacia composta da due postazioni fisse e una stampante.
In riferimento anche allo standard di cablaggio IEC 11801, realizzare la documentazione di rete (minisito o
word) indicando:
\begin{itemize}
	\item La topologia fisica applicata.
	\item I dispositivi utilizzati e le loro caratteristiche tecniche
	\item I mezzi di trasmissione utilizzati sia per il cablaggio verticale che orizzontale e le loro caratteristiche tecniche.
	\item Il tipo di connessione ad internet adottata e le sue caratteristiche. Il tipo di connessione deve essere scelta tra quelle viste (ADSL, FTTC, FTTB, FTTH) che meglio si adatta allo scenario.
	\item Il piano di indirizzamento
	\item Disegno di rete indicando una configurazione dell’access point, dei server e dei pc.
	\item Inoltre l’ospedale si vuole estendere aprendo una rete di ambulatori sul territorio cittadino. Ciascun ambulatorio sarà costituito da un ufficio amministrativo (1 pc e una stampante) e da una sala per la visita del paziente (1 pc). Ciascun ambulatorio deve interfacciarsi con la sede centrale. Per realizzare il disegno su packet tracer considerare due ambulatori collegati al router dell’ospedale mediante linea seriale. Indicare la tabella di instradamento dei tre router.
\end{itemize}

Facendo riferimento alla struttura del solo ospedale e volendo realizzare per ogni reparto inclusa la
farmacia una vlan e che solo la farmacia possa collegarsi ad internet, come cambia la struttura della rete?
Indicare la diversa configurazione.