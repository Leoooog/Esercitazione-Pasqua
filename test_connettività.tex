\section{Test di connettività}
\hspace{24pt}Infine si effettuano dei test di connettività tra i dispositivi. Si testa la connettività tra una
postazione dell'ospedale e il server Web prima tramite indirizzo IP e poi tramite il dominio "portaleinterno.it".
Il comando utilizzato sarà \mintinline{batch}{ping <destinazione>}. Da PC2 si lanciano i seguenti comandi:
\begin{itemize}
	\item \begin{minted}{batch}
	ping 192.168.0.3
	\end{minted}
	\item \begin{minted}{batch}
	ping portaleinterno.it
	\end{minted}
\end{itemize}
Il risultato è il seguente:
\vspace{10pt}
\begin{center}
	\makebox[\textwidth]{\includegraphics[width=\paperwidth-80pt]{test_connettività1}}
\end{center}
\vspace{10pt}
Si può osservare che anche nel secondo caso la destinazione è il server web, perché PC2 interroga prima il server DNS,
e dopo aver ricevuto l'indirizzo IP corrispondente al dominio interroga il server web.\\
\hspace{24pt}Si effettua un altro test per verificare la connettività degli ambulatori. In questo caso PC15 eseguirà
gli stessi comandi lanciati prima da PC2. Ecco il risultato:
\vspace{10pt}
\begin{center}
	\makebox[\textwidth]{\includegraphics[width=\paperwidth-80pt]{test_connettività2}}
\end{center}
\vspace{10pt}
Anche in questo caso la connettività è verificata.