\section{Piano di indirizzamento e tabella di routing}
\hspace{24pt}Si procede alla stesura del piano di indirizzamento.\\
L'ospedale avrà come indirizzo di rete 192.168.0.0/24. L'interfaccia interna del router dell'ospedale (default gateway) avrà come indirizzo IP 192.168.0.1.
A tutti gli host connessi alla rete interna dell'ospedale sarà assegnato un indirizzo IP in modalità dinamica, tramite un
apposito server DHCP locato nella sala server (192.168.0.2), ad eccezione del server web e del server DNS, il cui indirizzamento
sarà statico (rispettivamente 192.168.0.3 e 192.168.0.4).\\
\hspace{24pt}I due ambulatori, che per comodità si chiamano 1 e 2, avranno gli indirizzi di rete 192.168.1.0/24 e 192.168.2.0/24 e
l'indirizzamento sarà statico data la poca quantità di host appartenenti alla rete. Oltre all'indirizzo IP dell'host 
bisognerà specificare anche il default gateway e il server DNS.\\
\hspace{24pt}Il router dell'ospedale è collegato ai router dei due ambulatori tramite interfacce seriali, a cui sono stati assegnati
indirizzi di classe A. Di seguito la tabella di indirizzamento dei router. \\
\vspace*{12pt}
\begin{center}
    \begin{tabular}{| c | c | c |}
        \hline
        Router       & Interfaccia & Indirizzo IP \\ \hline
        Ospedale     & FA0/0       & 192.168.0.1  \\  \hline
        Ospedale     & SE0/0/0     & 20.0.0.1     \\ \hline
        Ospedale     & SE0/0/1     & 10.0.0.1     \\ \hline
        Ambulatorio1 & FA0/0       & 192.168.1.1  \\ \hline
        Ambulatorio1 & SE0/0/0     & 10.0.0.2     \\ \hline
        Ambulatorio2 & FA0/0       & 192.168.2.1  \\ \hline
        Ambulatorio2 & SE0/0/0     & 20.0.0.2     \\ \hline
    \end{tabular}
\end{center}

\vspace{12pt}
Per far comunicare i due ambulatori con l'ospedale bisogna configurare le tabelle di routing dei router. Si assume che
i due ambulatori non debbano comunicare tra di loro. Di seguito le tabelle per ogni router.\\
\vspace{12pt}
Ospedale:
\begin{center}
    \begin{tabular}{| c | c | c |}
        \hline
        Destinazione & Subnet Mask   & Next Hop \\ \hline
        192.168.1.0  & 255.255.255.0 & 10.0.0.2 \\ \hline
        192.168.2.0  & 255.255.255.0 & 20.0.0.2 \\ \hline
    \end{tabular}
\end{center}
\vspace*{12pt}
Ambulatorio1:
\begin{center}
    \begin{tabular}{| c | c | c |}
        \hline
        Destinazione & Subnet Mask   & Next Hop \\ \hline
        192.168.0.0  & 255.255.255.0 & 10.0.0.1 \\ \hline
    \end{tabular}
\end{center}
\vspace*{12pt}
Ambulatorio2:
\begin{center}
    \begin{tabular}{| c | c | c |}
        \hline
        Destinazione & Subnet Mask   & Next Hop \\ \hline
        192.168.0.0  & 255.255.255.0 & 10.0.0.2 \\ \hline
    \end{tabular}
\end{center}
\vspace*{12pt}
